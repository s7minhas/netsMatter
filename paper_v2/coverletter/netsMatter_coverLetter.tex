\documentclass[letterpaper]{article}
\usepackage{graphicx,fullpage}
\usepackage{hyperref}
\usepackage{geometry}
\usepackage[T1]{fontenc}
\usepackage[sc,osf]{mathpazo}

\geometry{
  body={6.5in, 8.5in},
  left=1.0in,
  top=1.25in
}
\usepackage{sectsty}
\sectionfont{\rmfamily\mdseries\Large}
\subsectionfont{\rmfamily\mdseries\itshape\large}
\setlength\parindent{0em}

% Make lists without bullets
\renewenvironment{itemize}{
  \begin{list}{}{
    \setlength{\leftmargin}{1.5em}
  }
}{
  \end{list}
}

\begin{document}
\thispagestyle{empty}

\begin{minipage}{0.64\linewidth}
Shahryar Minhas \\
Department of Political Science \\
Michigan State University University \\
368 Farm Lane \\
East Lansing, MI 48823\\
United States
\end{minipage}
\begin{minipage}{0.45\linewidth}
  \begin{tabular}{lr}
    Phone: & (248) 675-7345 \\
    Email: & \href{mailto:s7.minhas@gmail.com}{\tt s7.minhas@gmail.com}  \\
    Website:& \href{http://s7minhas.com/}{\tt s7minhas.com}
  \end{tabular}
\end{minipage}

\vspace{.5in}

{Editorial Team of Political Science Research \& Methods via submission portal}

\vspace{0.25in}

Dear Colleagues:\\[1ex]

This letter accompanies our submission of a manuscript for your consideration. The manuscript ``Taking Dyads Seriously'' discusses and tests the utility of the Additive and Multiplicative Effects (AME) framework for conducting inference in the context of the dependencies that often occur in dyadic data. In submitting this paper, we feel as though it is necessary to first provide some background information. \\[1ex]

Among some political scientists using statistical models to analyze network data there has emerged a dispute between the advocates of the Exponential Random Graph (ERGM) model and those building on the ``latent space'' approach pioneered by Hoff, Raftery, and Handcock. This unproductive fight is not mirrored in the statistical or broader networks literature, as in those communities people recognize these models as having quite distinct goals. These models having differing goals is something that I also believe and which is why in a recent ERGM related article I reviewed for PSRM, I did not express any need for the authors to engage in a lengthy contrast of their approach with latent variable models. \\[1ex] 

In political science, however, the dispute has become quite prevalent. Although we clearly work in the latent space world, this paper is not about engaging or extending the ERGM-latent space discussion. A Political Analysis piece by Minhas, Hoff, and Ward published this year already delves into the differences between various network approaches. This piece, however, is meant to evaluate the utility of the AME framework using a novel simulation based exercise and a replication of three recently published works from the field of International Relations. Through the simulation exercise, we show that in the presence of unobserved dependencies the model is able to provide less biased estimates and better calibrated standard errors than extant approaches in the literature. Next, for nearly all of the replications reestimated using the AME framework we find that the case for the key findings from the original works become much less compelling. Additionally, we show how the AME can actually be used by IR scholars to glain meaningful insights from the dependencies that estimated by the model.\\[1ex]

We believe that the AME framework is of notable interest to political scientists broadly, and that this study provides not only an introduction but a strong case for its applicability to the questions that we are seeking to address. We look forward to your evaluation of this paper.\\[1ex]

Respectfully submitted,

\vspace{.1in}

\includegraphics [scale=.6]{/Volumes/Samsung_X5/Dropbox/Finances/signature.png}

% \vskip 0.5in
% \hrule

\end{document}\bye
