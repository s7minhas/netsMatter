\section{\textbf{Introduction}}

\citet{aronow:etal:2015} claim that over the period from 2010 to 2015, over sixty articles utilizing dyadic data were published in the \textit{American Political Science Review}, \textit{American Journal of Political Science}, and \textit{International Organization}. Most of these studies  use a generalized linear model (GLM) to estimate regression coefficients.  However, extant approaches to studying dyadic data increase the chance of faulty inferences by treating data as independent and identically distributed (iid) when observations may be highly dependent.\footnote{Aronow et alia (2015) provide a solution to simple clustering via covariance estimation but do not address the broader issues of interdependence focused on herein.} Most standard approaches assume that the problems raised by having non-iid relational data can be addressed by recalculating the standard errors of estimated parameters in the link function, in order to reflect the potential clustering of cases. This may work in limited situations, but is not generally effective because these palliatives do not address the fundamental data generating process which remains a threat to inference because of the interdependence of observations or measurements. It is not just the diagonals of the variance-covariance matrix this affects.

In this article, we introduce a Bayesian approach with additive and multiplicative effects for directly modeling relational data to reflect the data generating process that yields interdependencies in relational data. We focus on three types of interdependencies that can complicate dyadic analyses. First, dependencies may arise within a set of dyads as there may be a particular actor that is more likely to send or receive events such as conflict. Additionally, if the event of interest has a clear sender and receiver, we are likely to observe dependencies within a dyad, specifically, if a rebel group initiates a conflict with a government, the government will reciprocate that conflictual behavior. We capture these two dependencies, referred to as first- and second-order dependencies within the additive effects portion of the model. The multiplicative effects portion of the model captures dependencies that results from groups of actors clustering together or organizing into communities. Clusters and communities often result in relational data because actors possess some set of shared attributes that affect their probability of interacting with one another. For example, when considering alliance patterns we might reasonably expect that if two actors are already allied to a third that their probability of allying with one another is higher than average. 

Third-order effects can arise from the presence of some set of shared attributes between actors that affects their probability of interact- ing with one another. Individuals try to close triads because this is thought to be a more stable or preferable social situation

a specific actor is responsible for sent or received events such 

This model has two parts, one an additive effects portion that 

, one of which uses a set of additive random effects to capture dependencies arising because a specific actor sent or received an events such as conflict to others. 

\first- and \second-order dependencies in the relational data. The second part is a multiplicative component that captures \third-order dependencies, i.e., phenomena that leads to triads such as homophily and stochastic equivalence.  We apply this model, called the Additive and Multiplicative effects (AME) model, to five prominent studies in the international relations literature, and compare results from the current state-of-the-art approach (a GLM with robust standard errors) to those obtained with the AME framework. 

This latent factor approach is able to better capture \first-, \second-, and \third-order interdependencies than the standard approaches. It also produces results that are more precise, and often different, than those found in these literatures, and as such offers substantive insights, which are occluded by ignoring the interdependent nature of the relational data that characterizes many studies in the field of international relations. Finally, we show that out-of-sample  predictions (cross-validations) are substantially improved with this approach.  