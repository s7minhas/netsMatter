\section{\textbf{Introduction}}

\citet{aronow:etal:2015} estimate that over the period from 2010 to 2015, over sixty articles utilizing dyadic data were published in the \textit{American Political Science Review}, \textit{American Journal of Political Science}, and \textit{International Organization}. Most of these studies use a generalized linear model (GLM) to estimate regression coefficients.  However, extant approaches to studying dyadic data increase the chance of faulty inferences by treating data as independent and identically distributed (iid) when observations may be highly dependent. Most standard approaches assume that the problems raised by having non-iid relational data can be addressed by recalculating the standard errors of estimated parameters in the link function, so as to reflect the potential clustering of cases. This may work in limited situations, but is not generally effective because these palliatives do not address the fundamental data generating process that remains a threat to inference because of the interdependence of observations or measurements. Namely, it is not just the diagonals of the variance-covariance matrix this affects. % \citep{beck:2012, king:roberts:2014}

In this article, we discuss a Bayesian approach, the Additive and Multiplicative Effects (AME) model, for directly modeling relational data to reflect the data generating process that yields interdependencies in these types of data structures \citep{hoff:2008,minhas:etal:2016:arxiv}. We focus on three types of interdependencies that can complicate dyadic analyses. First, dependencies may arise within a set of dyads as there may be a particular actor that is more likely to send or receive events such as conflict. Additionally, if the event of interest has a clear sender and receiver, we are likely to observe dependencies within a dyad; specifically, if a rebel group initiates a conflict with a government, the government will likely reciprocate that conflictual behavior. We capture these two dependencies, often referred to as first- and second-order dependencies, respectively, within the additive effects portion of the model. The multiplicative effects capture dependencies that results from groups of actors clustering together or organizing into communities due to \textit{meso-scopic} features of networks, such as homophily and stochastic equivalence. These type of meso-scopic features often arise in relational data because actors possess some latent set of shared attributes that affect their probability of interacting with one another. 

We begin by providing a brief review of these dependencies and the AME model. Next, we move to conducting a simulation study to show how this AME approach can recover unbiased and well-calibrated regression coefficients in the presence of network dependencies. Then, we apply this approach to five prominent studies in the international relations (IR) literature and compare results from the current state-of-the-art approach (a GLM with robust standard errors) to those obtained with the AME framework. The latent factor approach (AME) is able to better capture first, second, and third-order interdependencies than the standard approach. It also produces results that are more precise and at times at odds with those found in these studies in particular, and the broader literatures from which they are drawn. As such, this approach offers substantive insights which are often occluded by ignoring the interdependent nature of the relational data that characterize many studies in the field of international relations. Finally, we show that for each replication our network based approach provides substantively more accurate out-of-sample predictions than the models used in the original studies. Thus, the AME approach is one that can be used by scholars in the field to not only generate substantive insights, but it also enables us to better model the data generating process behind events of interest in international relations.  It facilitates the concentration on international relations in the field of international relations.