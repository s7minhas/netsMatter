\section{\textbf{Introduction}}

The aim of this study is to address how to estimate regression coefficients in a generalized linear model (GLM) context when there are network dependencies in dyadic data. Specifically, we discuss and evaluate how well the Additive and Multiplicative Effects (AME) model can be used to account for the interdependencies underlying the data generating process of dyadic structures \citep{hoff:2005,hoff:2008,minhas:etal:2019}. The AME works by including a set of parameters meant to capture network effects in the conditional mean equation of the GLM.

We focus on three types of network effects that can complicate dyadic analyses. First, dependencies may arise within a set of dyads if a particular actor is more likely to send or receive actions such as conflict.\footnote{In the case of undirected data where there is no clear sender or receiver, it is still essential to take into account the variance in how active actors are in the system.} Additionally, if the event of interest has a clear sender and receiver, we are likely to observe dependencies within a dyad; for example, if a rebel group initiates a conflict against a government, the government will likely reciprocate that behavior. We capture these effects, often referred to as first- and second-order dependencies, respectively, within the additive effects portion of the model. Third-order dependencies capture relationships of transitivity, balance, and clusterability between different dyads. For example, we can only understand why Poland was involved in a dyadic conflict with Iraq in 2003 if we understand that the United States invaded Iraq in 2003 and that Poland often participates in US-led coalitions. The multiplicative effects capture these sorts of dependencies, specifically, those that result because the specified model has not accounted for a latent set of shared attributes that affect actors' probability of interacting with one another.

We begin with a discussion of these dependencies and an introduction to the AME model. Next, we conduct a simulation study to show how the AME approach can recover unbiased and well-calibrated regression coefficients in the presence of network based dependence. Last, to highlight the utility of this approach, we apply the AME model to three recent studies in the international relations (IR) literature. Our comparison reveals that by accounting for observational dependence, leads to results that, at times, differ from those found in the original study as well as from the broader literature. Moreover, we demonstrate that the additional parameters included by AME in the conditional mean equation of a typical GLM can offer substantive insights that are often occluded by ignoring the interdependencies found in relational data. Finally, we show that for each replication our network-based approach provides substantially more accurate out-of-sample predictions than the models used in the original studies.

The AME approach advances statistical analysis of dyadic data by accounting for observational dependence while allowing scholars to test the substantive effect of variables of interest. Thus, the AME allows scholars to achieve a two-fold goal: to continue to generate meaningful, substantive insights about political phenomena without abandoning a regression based approach, while at the same time accounting for the data generating processes behind such events of interest. Most importantly, the AME approach concentrates on the relational aspect of the field of international relations through a statistical framework that is familiar to most scholars.

% \citet{aronow:etal:2015} estimate that during the period 2010 to 2015, over sixty articles were published in the \textit{American Political Science Review}, \textit{American Journal of Political Science}, and \textit{International Organization} using dyadic data.\footnote{In 2017, \textit{International Studies Quarterly} published a special issue on Dyadic Research Designs along with an online symposium to discuss the papers.} Most of these studies use a generalized linear model (GLM).  However, this approach to studying dyadic data increases the chance of faulty inferences by assuming data are conditionally independent and identically distributed (iid). Most standard approaches assume that the problems raised by non-iid relational data can be addressed by recalculating the standard errors of estimated parameters to reflect the potential clustering of cases. In practice, such strategies remain insufficient because they do not directly model underlying structural features in the data\footnote{As \citet[p.15]{aronow:etal:2015} explain, ``Of course, accounting for dynamic dyadic clustering may fail to fully account for all relevant dependencies in the data.''} but instead adjust the variance estimator.\footnote{While this approach decreases the risk of a type 1 error, it increases risk of type 2 error (if the mean estimate is biased downwards, increasing the variance estimate will cause us to falsely accept the null).} The aims of our study are twofold. First, we provide a framework that accounts for dependencies often present in dyadic data.\footnote{This is important to consider since the inferential problems caused by non-iid observations affect more than just diagonals of the variance co-variance matrix \citep{beck:2012,franzese:hays:2007,king:roberts:2014}.} Second, the framework we discuss here enables applied scholars to understand the structure underlying dyadic data rather than simply controlling for them.
