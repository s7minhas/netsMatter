\section{\textbf{Introduction}}

\citet{aronow:etal:2015} estimate that during the period 2010 to 2015, over sixty articles utilizing dyadic data were published in the \textit{American Political Science Review}, \textit{American Journal of Political Science}, and \textit{International Organization}. Most of these studies use a generalized linear model (GLM) to estimate regression coefficients.  However, this approach to studying dyadic data increases the chance of faulty inferences by assuming data are conditionally independent and identically distributed (iid). Most standard approaches assume that the problems raised by having non-iid relational data can be addressed by recalculating the standard errors of estimated parameters in the link function, so as to reflect the potential clustering of cases. In practice, these palliatves rarely work because they fail to address the fundamental data generating process that remains a threat to inference. Namely, it is not just the diagonals of the variance-covariance matrix that are affected \citep{beck:2012, king:roberts:2014}.

In this article, we discuss the Additive and Multiplicative Effects (AME) model, a Bayesian approach for directly modeling relational data to reflect the data generating process that yields interdependencies in dyadic data structures \citep{hoff:2008,minhas:etal:2016:arxiv}. We focus on three types of interdependencies that can complicate dyadic analyses. First, dependencies may arise within a set of dyads if a particular actor is more likely to send or receive events such as conflict. Additionally, if the event of interest has a clear sender and receiver, we are likely to observe dependencies within a dyad; for example, if a rebel group initiates a conflict with a government, the government will likely reciprocate that behavior. We capture these two dependencies, often referred to as first- and second-order dependencies, respectively, within the additive effects portion of the model. The multiplicative effects capture dependencies that result from groups of actors clustering together or organizing into communities due to \textit{meso-scopic} features of networks, such as homophily and stochastic equivalence. These type of meso-scopic features often arise in relational data because actors possess some latent set of shared attributes that affect their probability of interacting with one another. 

We begin with a brief review of these dependencies and the AME model. Next, we conduct a simulation study to show how the AME approach can recover unbiased and well-calibrated regression coefficients in the presence of the dependencies that arise in dyadic data. Then, we apply this approach to five prominent studies in the international relations (IR) literature and compare results from the current state-of-the-art approach (a GLM with robust standard errors) to those obtained using the AME framework. The comparison reveals that in ignoring observational dependence, the standard approach overestimates the effect of key variables in the literature and underestimates their uncertainty. Consequently, the latent factor approach (AME) produces results that, at times, contradict those found in these studies in particular, and the broader literatures from which they are drawn. Moreover, we demonstrate the latent factor approach offers substantive insights that are often occluded by ignoring the interdependencies in most relational data that the field of IR is concerned with. Finally, we show that for each replication our network based approach provides substantively more accurate out-of-sample predictions than the models used in the original studies. Thus, the AME approach can be used by scholars in the field to not only generate substantive insights, but it also enables us to better model the data generating process behind events of interest.  Most importantly, it facilitates concentration on the relations aspect of the field of international relations.