\section{\textbf{Applications with AME}}

\subsection{Design}

We apply AME to four recent IR studies: \citet{reiter:stam:2003, mcdonald:2004,  weeks:2012, gibler:2017}. Each of these studies are representative of broader trends in the field in that they use relational data of state interactions and propose both dyadic, monadic, and structural explanations for behavior of actors in the system. We choose to demonstrate the capabilities of AME with reference to existing studies in order to highlight several features of the AME approach. First, the results of AME estimation are interpretable alongside results using standard approaches, but, as shown in the simulation section, have the additional benefit of being able to take into account dependencies that may complicate inference. Second, through using this approach, we can also quantify the degree to which first, second, and third order dependencies are present in events of interest. Third, we show that by using the AME framework scholars can better model the data generating process behind their events of interest.

We obtained the data for each of these studies from their replication archives and replicated the main results of each article.\footnote{Without exception this was straightforward to accomplish, thanks to the authors' transparency and an increasing norm in the social sciences of open data sharing.} The four studies used below were selected based on how recent they were and whether they had more then 100 citations.\footnote{Note that we chose papers with at least 100 citations as an indicator of influence within the discipline. Of course, by the criteria of influence many other papers could have been considered for the replication. However, as described in the following sections, the AME model has specific data requirements, which limited the scope of potential papers.} Each of these pieces, published in prominent journals well-known in their respective literatures, posited a theory in which interdependencies are consequential. Reflecting the dominant approach in the literature, each of the authors tested their hypothesis by employing some form of a general linearized model.\footnote{It is important to note that the AME framework is applicable only where a full set of dyads for an outcome of interest is observable---for example it is unsuitable for studying onsets, where dyad-years representing conflict continuations are removed from the sample. The AME is also not applicable to analyses of case-level data, for example, studies that examine the decision to go to war by particular states.}

\begin{table}
\caption{Features of the Studies Re-estimated.}
	\begin{tabular}{lcccccc}
		& Model &  Date Range & N. Actors  & N. Dyads & Dyads Type & Clustering $\sigma_{\hat{\beta}}$ \\ \toprule
		Reiter \& Stam (2003) &Logit &1945--1995 &  193 & $753,456$ & Directed & Robust \\	
		McDonald (2004) & Logit &1959--2002 & 198 & $92,354$ & Undirected & Robust\\
		% Rose (2004) & OLS & 1948--1999 & 177 & $234,597$ & Directed & Robust \\	 
		Weeks (2012) & Logit & 1946--1999 &197 &  $901,540$ & Directed & Robust \\
		Gibler (2017) & Logit & 1816--2008 &193 &   $650,557$ & Undirected & None \\ \bottomrule
	\end{tabular}
\end{table}

Each of the four studies has a crucial finding that we hone in on to further draw into focus the analytical power of the AME estimation procedure.  In Table~\ref{tab:modelFindingSumm}, we present the overall results; the term \textit{Unconfirmed} indicates only that the sign and/or significance of the putatively crucial finding in the original study is not found to hold in the AME estimation.\footnote{Full tabular results for each of the original and reestimated models are presented in the Appendix.}

\begin{table}[ht]
\centering
\caption{Here we provide a brief summary of the key variable in each of the four replications and a note about whether or not the highlighted finding remains when using our network-based approach.}
	\begin{tabular}{l p{7cm} l} \toprule
		\multirow{2}{*}{Study} & \multirow{2}{*}{Central Finding} &  Confirmed after \\
		& &  accounting for dependencies? \\ \toprule
		Reiter \& Stam (2003) & Personalist Regimes Attack Democracies, Not Vice Versa & {Partially Confirmed} \\ \midrule
		McDonald (2004) & Lower Trade Barriers and Higher Trade Lead to Peace & {Confirmed}\\ \midrule
		% Rose (2004) & WTO Membership Does not Affect Trade & {Partially Confirmed}\\ \midrule
		Weeks (2012) & Bosses, Juntas, and Strongmen are more Aggressive, Machines are Not & {Unconfirmed} \\\midrule
		Gibler (2017) & Power Parity at Time of Entry to International System Increases Conflict & {Unconfirmed}\\ \bottomrule
	\end{tabular}
	\label{tab:modelFindingSumm}
\end{table}

An important takeaway here is that many scholars are forced to make knowledge claims based on the statistical significance of a small set of covariates, or the differences between these covariates. These differences may change dramatically when interdependencies are taken into account directly. This outcome follows from AME's ability to better account for the dependencies discussed in the previous section, whereas GLM approaches explicitly assume observational independence conditional on the specified covariates. As this is a widely-known limitation of GLM approaches, scholars often attempt to account for clustering of observations by including additional variables and adjusting the standard errors of the resulting estimates. At best, this method introduces noise and imprecision into results, and at worst can produce misleading outcomes. 

Beyond just comparing parameter estimates, we examine how well each approach can represent the data generating process using an out-of-sample cross validation strategy. Specifically, for each study, we randomly divide the data into $k=30$ sets, letting $s_{ij,t}$ be the set to which pair $ij,t$ is assigned.

Then for each $s \in \{1,\ldots,k\}$, we:

\begin{enumerate}
	\item estimate model parameters with $\{y_{ij,t}: s_{ij,t} \neq s\}$, the data not in set $s$,
	\item and predict $\{\hat{y}_{ij,t}: s_{ij,t} = s\}$ from these estimated parameters. 
\end{enumerate}

The result of this procedure is a set of sociomatrices $\bm \hat Y$, in which each entry $\hat y_{ij,t}$ is a predicted value obtained from using a subset of the data that does not include $y_{ij,t}$. We summarize the performance of the various models in Table~\ref{tab:modelPerfSumm} below. For the binary models we provide the area under the Receiver Operator Characteristic (ROC) and Precision Recall (PR) curves. Only one of the studies here had a continuous dependent variable and for this we provide the root mean squared error (RMSE) and root median squared error (RMDSE).\footnote{More details on the performance of each of these models can be found in the Appendix.} For each of the replications, we find that the AME approach substantially outperforms the original models in terms of out-of-sample predictive performance. This is important as it indicates that switching to the AME framework---even when using the exact same specification as the original studies---enables scholars to better represent the data generating process of their events of interest. The fact that this analysis is done in an out-of-sample context ensures that the AME framework is not simply overfitting with more parameters, rather the dependence parameters we include are capturing underlying structure previously missed by the exogenous covariates in the models.

\begin{table}[ht]
\centering
	\begin{tabular}{l|l cc}
	~ & ~ & GLM & AME \\
	\toprule
		\multirow{2}{*}{Reiter \& Stam (2003)} & Area Under ROC Curve, AUC-ROC & 0.92 & {0.96} \\
				~ & Area Under PR Curve, AUC-PR & 0.08 &  {0.15} \\		\midrule
		\multirow{2}{*}{McDonald (2004)} & AUC-ROC & 0.92 &  {0.99} \\
				~ & AUC-PR & 0.13 &  {0.28} \\		\midrule
		% \multirow{2}{*}{Rose (2004)} & RMSE & 3.23 &  {1.99} \\
		% 		~ & RMDSE & 2.01 &  {1.06} \\	\midrule
		\multirow{2}{*}{Weeks (2012)} & AUC-ROC & 0.64 &  {0.97} \\
				~ & AUC-PR & 0.00 &  {0.15} \\		\midrule
		\multirow{2}{*}{Gibler (2017)} & AUC-ROC & 0.52 &  {0.91} \\
				~ & AUC-PR & 0.00 &  {0.08} \\			\bottomrule
	\end{tabular}
	\caption{Here we provide a summary of the out-of-sample performance based on our cross-validation strategy for each of the four replications when using the standard dyadic approach and our network-based approach. Four of the four studies involved a binary dependent variable; for those measures, area under the curve (AUC) statistics are reported.\tabularnewline
	%The Rose study involved a Gaussian dependent variable and for that we use the root mean squared error (RMSE) and root median squared error (RMDSE). 
	}
	\label{tab:modelPerfSumm}
\end{table}
\FloatBarrier

