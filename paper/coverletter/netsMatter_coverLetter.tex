\documentclass[letterpaper]{article}
\usepackage{graphicx,fullpage}
\usepackage{hyperref}
\usepackage{geometry}
\usepackage[T1]{fontenc}
\usepackage[sc,osf]{mathpazo}

\geometry{
  body={6.5in, 8.5in},
  left=1.0in,
  top=1.25in
}
\usepackage{sectsty}
\sectionfont{\rmfamily\mdseries\Large}
\subsectionfont{\rmfamily\mdseries\itshape\large}
\setlength\parindent{0em}

% Make lists without bullets
\renewenvironment{itemize}{
  \begin{list}{}{
    \setlength{\leftmargin}{1.5em}
  }
}{
  \end{list}
}


\begin{document}
\thispagestyle{empty}

  
  
  \begin{minipage}{0.64\linewidth}
Shahryar Minhas \\
Department of Political Science \\
Michigan State University University \\
368 Farm Lane \\
East Lansing, MI 48823\\
United States
\end{minipage}
\begin{minipage}{0.45\linewidth}
  \begin{tabular}{lr}
    Phone: & (248) 675-7345 \\
    Email: & \href{mailto:s7.minhas@gmail.com}{\tt s7.minhas@gmail.com}  \\
    Website:& \href{http://s7minhas.com/}{\tt s7minhas.com}
  \end{tabular}
\end{minipage}
  
\vspace{1.5in}

{Editorial Team of Political Analysis via submission portal}

\vspace{0.5in}

Dear Colleagues:\\[1ex]

This letter accompanies our submission of a manuscript for your consideration. The manuscript ``Taking Dyads Seriously'' introduces the Additive and Multiplicative Effects (AME) framework for conducting inference in the context of the dependencies that we often observe in dyadic data.\\[1ex]

We evaluate the utility of the AME framework using a simulation based exercise and a replication of five recently published works from the field of International Relations. Through the simulation exercise, we show that in the presence of unobserved dependencies the model is able to still provide less biased estimates and better calibrated standard errors than extant approaches in the literature. Next, for almost each of the replications reestimated using the AME framework we find that the case for the key findings from the original works become much less compelling. Additionally, we show through an out-of-sample cross-validation exercise that the AME approach uniformly outperforms each of the replicated approaches in capturing the data generating process of the event of interest.\\[1ex] 

We believe that the AME framework is of notable interest to political scientists broadly, and that this study provides not only an introduction but a strong case for its applicability to the questions that we are seeking to address.\\[1ex]

We look forward to your evaluation of this paper.\\[1ex]

Respectfully submitted,

\vspace{.1in}

\includegraphics [scale=.8]{/Users/s7m/Dropbox/Finances/signature.png}

% \vskip 0.5in
% \hrule

\end{document}\bye
