\section{Conclusion}

International relations is generally about the interactions and dependences among a set of countries or other important actors such as international governmental organizations (IGOs, such as the WTO) and non-governmental organizations (NGOs, such as the Gates Foundation).  Some approaches focus on only looking a at small number of these actors, but many scholars examine a large number of actors at a time. This is particularly true of those scholars who work in the tradition of the Correlates of War Project, but is by no means limited to them.\footnote{See \cite{singer:1972} for an early description of the project and also see the project's Web site for an history and more recent efforts \url{http://www.correlatesofwar.org/}.} Many scholars have debated the use and abuse of dyadic data. One recent on-line symposium can be found at \url{http://bit.ly/2wB2hab}. It is clear from a survey of the literature and from work in this area published as recently as 2017 that many find dyadic data to be an important touchstone in the study of international relations \citet{erikson:pinto:2014,aronow:etal:2015}.

At the same time, we know that research designs which focus on the statistical analysis of dyadic data quickly go astray if the dyadic data are assumed to be independent and identically distributed (iid).  Virtually all of the standard statistical models---ordinary least squares, logistic and probit regressions, to name a few---fail if the data are not iid. By definition dyadic data are not iid and thus the standard approaches can not be used cavalierly to analyze these data.  \citet{signorino:1999} showed why this is true of models of strategic interaction, but it is more broadly true of models that employ dyadic data.  We show that latent networks can be employed to defeat this vulnerability of dyadic data in the realm of international relations. These approaches have been developed for a while, but are not yet widely used in international relations scholarship.  The statistical model of the latent network captures first-order (example or restatement), second-order (example or restatement), and third-order (example or restatement) dependencies in dyadic data using a familiar regression framework that has been adapted for relational data---such as dyadic data---which are not independent nor identically distributed. There is an available, open source computer packages that implements this approach.

To explore this approach in the context of international relations we conduct two broad analyses. The first is a simulation of where the characteristics of the network are known. This shows that the AME approach is unbiased in terms of parameter estimation compared with standard approach employed in
international relations to study dyadic data (i.e., GLM models). The second is a replication of five prominent studies that have been published recently using a broad range of dyadic data to draw inferences about international relations.  These five studies have been replicated with the original research designs, each of which used a statistical method that assumes the dyadic data are all independent from one another.  We then reanalyzed each study using the latent network approach which captures that additive and multiplicative aspects of interdependencies among the dyadic data.  In every case, we found that the AME approach provided a) increased precision of estimation, b) better out-of-sample fits, c) evidence of 1st-, 2nd-, and 3rd-order dependencies that were overlooked in the original studies.\footnote{The Appendix contains performance data on all of these replications, as well as sample code illustrating how to undertake AME analysis using \texttt{amen}.} In several cases, the new approach overturns the basic findings of the original research.  This leads us to speculate that many of the findings in the international relations literature may be fragile in the sense that they only obtain under stringent assumptions that can not possibly be valid.  This in turn leads to a certain arbitrariness in some research findings, which might lead to puzzles that are more apparent than real \citep{zinnes:1980}.  At the same time, the latent factor model provides a way to easily examine and if necessary defeat these assumptions.

It is no longer necessary to assume that the interesting, innate interdependencies in relational data about international relations can be ignored. Nor do they have to be approximated with \textit{ad hoc}, incomplete solutions that purport to control for dependencies (such as modifying the post-estimation standard errors of the estimated coefficients \citep{king:roberts:2014}). Instead, the interdependencies may be addressed directly with additive and multiplicative effects in the context of a generalized linear model that provides more reliable inferences and better out-of-sample predictive performance, along with new substantive insights. 