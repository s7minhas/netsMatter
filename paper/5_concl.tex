\section{\textbf{Conclusion}}

International relations is about the interactions, relationships, and dependencies among countries or other important international actors. This is particularly true of scholarship in the tradition of the Correlates of War Project, but it is by no means limited to it.\footnote{See \cite{singer:1972} for an early description of the project and also see the project's Web site for an history and more recent efforts \url{http://www.correlatesofwar.org/}.} Many scholars have debated the use and abuse of dyadic data.\footnote{One recent on-line symposium can be found at \url{http://bit.ly/2wB2hab}.} A broad survey of the IR literature makes it clear that scholars find dyadic data to be an essential touchstone in the study of international relations \citep{erikson:pinto:2014,aronow:etal:2015}. Our findings bolster a growing recognition in the field of International Relations that interdependence influences not only statistical estimations, but how scholars theorize about internationally relevant politics. Scholars have demonstrated the theoretical importance of interdependence through research on intrastate conflict \citep{dorff:etal:2020}, interstate bargaining \citep{gallop:2017}, economic interdependence \citep{maoz:2009a}, and international treaties \citep{kinne:2013} among other topics.

At the same time, we know that research designs focusing on the statistical analysis of dyadic data quickly go astray if the dyadic data are assumed to be iid.  Virtually all of the standard statistical models---ordinary least squares and logistic regressions, to name a few---fail if the data are not conditionally independent. This fact has been accepted as it relates to temporal dependencies, but adoption of methods to account for network dependencies have seen less progress. By definition dyadic data are not iid and thus the standard approaches cannot be used cavalierly to analyze these data. \citet{signorino:1999} showed why this is true of models of strategic interaction, but it is more broadly true of models that employ dyadic data.  We show that the AME framework can be employed to account for the statistical issues that arise when studying dyadic data.

To explore this approach in the context of international relations we have presented two analyses. The first is a simulation where the characteristics of the network are known. This shows that when there are unobserved dependencies, the AME approach is less biased in terms of parameter estimation compared to the standard approach employed in international relations to study dyadic data (i.e., GLM models). The second analysis is a replication of recent studies that use a broad range of dyadic data to draw inferences about international relations.  These studies have been replicated with the original research designs, each of which used a statistical method that assumes the dyadic data are all independent from one another.  We then re-analyzed each study using the AME model.  In every case, we found that the AME approach provided a) increased precision of estimation, b) better out-of-sample fit, and c) evidence of 1st-, 2nd-, and 3rd-order dependencies that were overlooked in the original studies.\footnote{The Appendix contains performance data on all of these replications, as well as sample code illustrating how to undertake AME analysis using \texttt{amen}.}

% It is no longer necessary to assume that the interesting, innate interdependencies in relational data about international relations can be ignored. Nor do they have to be approximated with \textit{ad hoc}, incomplete solutions that purport to control for dependencies (such as modifying the post-estimation standard errors of the estimated coefficients \citealp{king:roberts:2014}). Instead, the interdependencies may be addressed directly with additive and multiplicative effects in the context of a generalized linear model that provides more reliable inferences, better out-of-sample predictive performance, and new substantive insights.
