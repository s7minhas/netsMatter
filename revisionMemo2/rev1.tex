\section*{Reviewer 1}

\begin{enumerate}
	\item In my previous review I pointed out some problems with vague language, mostly in the opening sections.   Honestly, I still find the introduction to be problematic. Maybe it could be cut and we just open with the part of section 2 where the paper begins to discuss processes that lead to dyadic and higher order dependencies?  The reason is that the intro tries to strike a very general tone.  But in reality, the paper proposes a specific estimation strategy for parameters in a specific parametric model of networked interactions.  I think the paper would be more useful to readers if it started with the specific applications, and then on the basis of that, motivated the statistical specification on substantive grounds.  There are some swipes at other literature are neither necessary nor clear enough to be very compelling.
	\begin{itemize}
		\item \emph{ \textcolor{blue}{
		We thank the reviewer for taking the time to provide their comments and helping us to make the manuscript more focused. We have made a number of changes to the framing of the article per the reviewer's suggestions. Additionally, we have significantly modified the introduction and removed what could be considered as, but certainly not intended, swipes at other literatures. The editor also suggested a number of changes to be made in the introduction and we have adopted those as well. We disagree with the choice of removing the introduction completely and starting with section 2 because we feel some relatively non-technical context will be useful to more general readers.
		}}
	\end{itemize}
	\item ``Most of these studies use a generalized linear model (GLM). However, this approach to studying dyadic data increases the chance of faulty inferences by assuming data are conditionally independent and identically distributed (iid).'' It should really read, ``Most of these studies use a generalized linear model (GLM). However, this approach to studying dyadic data increases the chance of faulty inferences if the estimation and inference assume data are conditionally independent and identically distributed (iid).''  But then if this is all just about mistakenly assuming iid, just say it quickly and move on.
	\begin{itemize}
		\item  \emph{ \textcolor{blue}{
		We thank R1 for this suggestion, per R1's first comment, we have actually removed this from the introduction and clarified the goal of the paper along the lines of what R1 and the editor suggested.
		}}
	\end{itemize}
	\item Why would the mean estimate be biased downward?  And, biased with respect to what target of inference?  A problem with the opening discussion is that it does not establish a target of inference.  Without doing so, ``bias'' is undefined.  It seems like they mean the coefficient for the linear relationship between some regressor and the latent outcome variable. But what if one is using a nonparametric model or a semiparametric model that operates on a different latent space---is bias even defined?  It seems that for bias to be defined we have to be restricting ourselves to a class of models that propose the same latent space, but that makes this much less general that the discussion leads on.  Moreover, that paper is not committing to talking about bias relative to causal effects under effect heterogeneity, but isn't that usually what people are after?  Or, maybe the paper is really concerned with bias with respect to a forecast prediction? If so, then we need to set up a forecast prediction error analysis.
	\begin{itemize}
		\item  \emph{ \textcolor{blue}{
		These are fair questions especially in terms of the complications that arise when using nonparametric or semiparametric models. Instead of trying to resolve all this in the introduction, we have removed the portion of the introduction that led the reviewer to ask this question.
		}}
	\end{itemize}
	\item Moving to section 2, the extended discussion of specifying a likelihood under iid is overkill.  Again, I would just move on to the discussion of the application. The point that iid is inappropriate will receive due attention in the course of developing the networked interaction model.
	\begin{itemize}
		\item  \emph{ \textcolor{blue}{
		We understand the position that a detailed discussion of specifying a likelihood under iid seems laborious and potentially unnecessary. After a lengthy discussion among our co-authors, we have decided to retain this section of the paper. We believe that since our approach is directly aimed at applied scholarship in International Relations, and because we focus on applications that have appeared in top journals during the last 10 years, this section enables our study to serve a broad audience of readers. By keeping it, we ensure that a reader can directly follow our main points about GLM specifications and how the AME addresses these concerns.
		}}
	\end{itemize}
	\item Explain how inference is done for the parameters from the AME model. How are intervals produced exactly? (Need this because you examine these in the simulation and make many statements about intervals throughout the text.)  What are the operating characteristics, in terms of convergence properties, etc.?
	\begin{itemize}
		\item  \emph{ \textcolor{blue}{
		Bayesian inference for the parameters from the AME model proceed via Gibbs sampling algorithms where we iteratively simulate values of the model parameters from their full conditional distributions. We have added additional discussion about how the parameters are estimated to both the simulation section and the applications. In Appendix A we provide a brief outline of the MCMC algorithm used. We do not re-derive the full conditional distributions for the AME models in this paper as it has been done in previous work that we have cited: Hoff 2005, 2008; Minhas et al. 2019. Additionally, Hoff (2021), which we now cite in the paper, more fully details the history of the AME model and how it has evolved since its original formulation in Hoff et al. (2002).
		}}
	\end{itemize}
	\item The DGP is simple enough that the paper needs to include the analytical calculation of the bias from the simple probit regression of Y on X.  We get no insight by just having monte carlo simulation results. For the simulations, clarify the ``standard international relations approach'' as just Y regressed on X, when you introduce it. Then explain how confidence intervals were produced. Are these just normal-approximation Wald intervals with non-robust standard errors? Explain exactly how the confidence intervals are produced for the AME model.  They were never characterized in the section that discussion the model.
	\begin{itemize}
		\item  \emph{ \textcolor{blue}{
		Given the various forms of network dependence that can arise it's not clear to us how to actually construct an analytical calculation like what R1 is suggesting. If the purpose of this analytical calculation would be to highlight the consequences that dyadic dependence would have in the context of inference, then that is exactly why we open Section 2 in the paper with a discussion of the assumptiong underlying GLMs. }}
		\item  \emph{ \textcolor{blue}{
		In terms of the simulation section, we thank R1 for pointing out the parts that are still unclear and have added in their suggested language with respect to clarifying the standard IR approach.
		}}
		\item  \emph{ \textcolor{blue}{
		We thank R1 for asking a question about how confidence intervals are produced because that was an error on our part as we meant to say credible interval. Within our framework These are generated directly from the posterior distribution generated from the three models. Specifically, for every simulation and model we calculate the 95\% credible interval and then across all simulations we estimate the proportion of times that the true value fell within those intervals. We have clarified this in the simulation section of the manuscript.}}
	\end{itemize}
\end{enumerate}
