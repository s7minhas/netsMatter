\section*{Reviewer 2}

\begin{enumerate}
	\item I would have liked to see a justification for the choice of papers used in the replications. It is still unclear. Are these cases where the theory fail to characterize the direct and indirect effects, and implicitly assume that the latent dependencies are nuisances?
	\begin{itemize}
		\item \textcolor{blue}{ \emph{
		Thank you for raising this concern. In footnote 20 we state the primary criteria for our choice of papers used: we focus on studies explicitly about International Relations, were published after the year 2000, and were published in a top ranking general political science journal (for consistency in editorial standards and reviews, we focus on one journal, the American Political Science Review). In addition to these criteria, we searched for studies we understood to have hypotheses in which interdependencies are theoretically consequential due to the dyadic setting in which their theories operate (p.14) but wherein dependencies are not directly modeled (beyond the use of standard error corrections). We chose papers that, in spirit, seem to treat interdependence as technical problem rather than a threat to inference within the GLM context. In this round of revisions, we note clearly our focus on the GLM context starting with the first sentence of the MS.}}
	\end{itemize}
	\item The presentation of the AME model clearly states that the Xs are exogenous; it is likely that they are not in replications. Still, the AME estimates (which in the simulations do not seem to outperform the standard method for estimating the betas when correlation between X and W is high) suggest that the models in the original papers are misspecified, and that these differences are likely driven by a better modeling of the DGP. This is a very important take-home point.
	\begin{itemize}
		\item \textcolor{blue}{ \emph{
			Thank you for highlighting this point. We hope that this take home point is now made clear throughout the paper, but especially in sections 5.4 `lessons learned' and in our conclusion. We have also added the following change to section 5.4 ``In each of the models the AME substantially outperforms the original model out of sample. This may be because by ignoring dependencies, the original models are misspecifying the DGP, and the AME better accounts for it."
			}}
	\end{itemize}
\end{enumerate}
