\section*{Editor}

\begin{enumerate}
	\item `effect'' is not precisely defined, so you might substitute ``how to estimate regression coefficients in a GLM context''
	\begin{itemize}
		\item \textcolor{blue}{ \emph{
		We thank you for this suggestion and have changed the manuscript accordingly, particularly in the first line of the manuscript.}}
	\end{itemize}
	\item Also, my understanding is that modeling the ``underlying structural features in the data'' means including these ``structural features'' in the conditional mean equation of the GLM rather than including them exclusively in the variance equation (or ignoring them altogether). Or, more precisely, including SRRM and LFM parameters in the conditional mean equation. The advances you make come from this specific modeling choice. Augmenting the GLM in this particular way leads to 1) unbiased (or nearly unbiased) estimates of regression coefficients under homopholic dependence, when these coefficients exist and are known (the Monte Carlo simulations) and 2) superior out-of-sample forecast predictions for real data, in which case the regression model is just an approximation with unknown parameters (the replication exercises). Including LFM parameters in the conditional mean equation also makes it more likely that IR scholars will treat dyadic dependence as substantively important.
	\begin{itemize}
		\item \textcolor{blue}{ \emph{
			Thank you for these suggestions. We have made significant modifications to the introduction in line with the recommendations that you have posted and those from reviewer 1. We have removed references that reviewer 1 perceived as swipes at previous work. In addition, we have adopted your helpful language about how the AME approach essentially calls for including structural features in the conditional mean equation of the GLM. 
			}}
	\end{itemize}
\end{enumerate}
