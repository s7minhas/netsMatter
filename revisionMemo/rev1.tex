\section*{Reviewer 1}

\begin{enumerate}
	\item P1: ``Most standard approaches assume that the problems raised by non-iid relational data can be addressed by recalculating the standard errors of estimated parameters to reflect the potential clustering of cases. In practice, such strategies rarely work because they do not directly address the fundamental data generating process.''
	\begin{itemize}
		\item The term “rarely work” is too casual.  What problem exactly is motivating this paper?  E.g., the Aronow et al. paper cited at the start of the same paragraph looks at consistent variance estimators for semiparametric regression estimators on dyadic data. By the standards set out by those authors in that paper, the strategies “work” just fine.  I suppose then that the current paper has a different sense of what it means for something “to work.” For example, I am supposing that the current paper is talking about people misconstruing the structural interpretation of a reduced form parameter estimated in a GLM fit to dyadic data.  That would be wholly orthogonal to what Aronow et al. were considering, but then this needs explication.
	\end{itemize}
	\begin{itemize}
		\item  \emph{ \textcolor{blue}{
		We thank the reviewer for taking the time to provide these valuable comments. We have clarified our language in the paper (on page 1) to address this question.  Strategies like those employed by Aronow et al. do not directly model all relevant dependencies in the data but instead adjust the variance estimator. As Aronow et al. explain, "Of course, accounting for dynamic dyadic clustering may fail to fully account for all relevant dependencies in the data" (p. 16).   While such an approach decreases the risk of a type 1 error, it increases risk of type 2 error (if the mean estimate is biased downwards, increasing the variance estimate will lead us to falsely accept the null). The main aim of our research is to directly account for the relevant dependency structures in the data, thus providing an accurate estimate and addressing both types of possible errors.
		}}
		%\item \textcolor{blue}{ \emph{
		%Aronow et al. also explicitly note that their approach does not deal with issues of third order dependence.
		%}}
		%\item \textcolor{blue}{ \emph{
		%Aronow et al. approach does not deal with bias in parameter estimates.
		%}}
	\end{itemize}
	\item P2: “Scholars working with dyadic data typically begin by stacking observations associated with each dyad on top of one another. This makes sense if each observation is independent of the others.”
	\begin{itemize}
		\item Revise this opening, since it is perfectly fine to stack observations and then incorporate auxiliary information (such as adjacency matrix) to characterize dependencies between rows.
	\end{itemize}
	\begin{itemize}
		\item {\emph{\textcolor{blue}{
		 The point is well taken, we have clarified this section (on page 2) to point to the assumption of independence as the primary problem, not the stacking of observations in and of itself.
		}}}
	\end{itemize}
	\item P3: “Unless one is able to develop a model that can account for the variety of explanations that may play a role in determining why a particular actor is more active than others, parameter estimates from standard statistical models will be biased.”
	\begin{itemize}
		\item This is another example of language that is too causal and imprecise, like the opening paragraph. What are “parameter estimates from standard statistical models”?  And “biased” with respect to what target of inference? Again, I think the authors have in mind “bias” with respect to parameters that are meant to have a particular structural interpretation, but this needs to be clarified.
		\item Generally, section 2 should be clearer in motivating the problem.  Provide us with a toy example that illustrates the kind of “bias” that the paper has in mind from “standard models”.  For me at least, after reading section 2 I still wasn’t sure what problem the paper is actually trying to address.
		\item \textcolor{blue}{ \emph{ We think the reviewer makes a good suggestion here to clarify, mathematically or in some other illustrative way, exactly how we think that dependencies bias the estimator. For example we can say that our target of inference is the estimation of beta (with some variance). If a coefficient is correlated with a dependency, then we could obtain an estimate for beta that is artificially high (and the same is true for the reverse).
		}}
		\item \textcolor{blue}{ \emph{ Additionally, we have added language motivating whatn is to come in the simulation section.
	}}
	\end{itemize}
\end{itemize}
	\item P7: Regarding the specification, this augments a dyadic random effects model, which is quite common, with  the LFM component, which resembles an interactive fixed effects specification (cf. Bai 2009, Ecta) and is based on an SVD factorization of the observed network.
	\begin{itemize}
		\item It is useful to think of the infeasible estimator and the implications for interpreting the LFM component.  The infeasible estimator takes the transformation f(.), fixed effects parameters, and SRRM random effects as known, in which case the LFM amounts to predictions from a factor model fit to the adjacency matrix of residuals on the scale of the latent outcome.  Would this be a reasonable interpretation? If so it would be useful to draw the connection to factor models commonly in use these days (such as the interactive fixed effects model of Bai, which is the basis of, e.g., generalized synthetic control methods).
	\end{itemize}
	\begin{itemize}
		\item \textcolor{blue}{ \emph{
		We thank the reviewer for providing this reference and have added discussion of this alternative to the manuscript.
		SM insert great response.
		paper by Bai: https://sci-hub.tw/https://onlinelibrary.wiley.com/doi/10.3982/ECTA6135
		connection between LFM and interactive fixed effects?
		}}
	\end{itemize}
	\item P8: “accounting for this structure is necessary if we are to adequately represent the data generating process.” Continuing with the problems I have with the language above, what does “adequately represent” mean?. P9: “a) biased estimates of the effect of independent variables, b) uncalibrated confidence intervals, and c) poor predictive performance.”
	\begin{itemize}
		\item Okay here is where the paper are starting to clarify goals.  So a) is about causal identification.  But as I understand, the solution here is parametric identification of causal effects.  This means that the issue of misspecification and robustness ought to be taken up.  Similarly, b) is addressed in this paper under a fairly simple parametric data generating process. But again, how robust is the proposed model to situations where the real DGP does not abide by the parametric assumptions (linearity, normally distributed random effects, and a factor model for the residuals of dimension K)? For c), I think most would agree that the “proof” is simply in comparison to whatever other good models are available.  Not however, that in social science applications, forecasting and therefore c) are rarely of interest (at least not in the types of research that appear in political science journals).  As such, from the perspective of what the researchers in discipline mostly cares about, a) and b) are of primary importance.
	\end{itemize}
	\begin{itemize}
		\item  \emph{\textcolor{blue}{
			For part (a), we have clarified that our study does not address causal identification. We instead focus on providing unbiased estimates of covariates, which improve inference for observational data. We have clarified the goals of the paper in the earlier sections (such as the opening paragraph of page 1) and have also edited P8 to read: ``Non-iid observations in relational data result from the fact that there is a complex structure underlying the dyadic events or processes that we observe." }}
		\item \emph{\textcolor{blue}{
			As for (b), the reviewer raises an important point. How well does AME work out when the DGP does not abide by parametric assumptions? We address this in the section concerning the simulation and Aronow's approach. }}
		\item \emph{\textcolor{blue}{Finally, we understand these goals are not always central to political science research, but we believe that prediction is both a valuable metric to evaluate model performance as well as a goal for scholars in and of itself, as in Gleditsch and Ward 2013; Grimmer 2015; Neunhoeffer and Sternberg 2019; Colaresi and Mahmood 2017; and Mueller and Rauh 2018 among others.}
		}
	\end{itemize}
	\item P9-10: we need to know exactly how and from what distributions X and W are being drawn to interpret the results here.  Are they independent/orthogonal or no?
	\begin{itemize}
		\item \textcolor{blue}{ \emph{
		SM will deal with this.
		}}
	\end{itemize}
	\item Section 4 needs to incorporate an assessment of robustness to misspecification.  See the Aronow et al. paper for an example of how to do this, where they study the performance of a dyadic random effects model.
	\begin{itemize}
		\item \textcolor{blue}{ \emph{
		SM will deal with this.
		}}
	\end{itemize}
\end{enumerate}
